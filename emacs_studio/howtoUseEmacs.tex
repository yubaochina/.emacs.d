% Created 2017-05-08 Mon 01:28
% Intended LaTeX compiler: pdflatex
\documentclass[11pt]{article}
\usepackage[utf8]{inputenc}
\usepackage[T1]{fontenc}
\usepackage{graphicx}
\usepackage{grffile}
\usepackage{longtable}
\usepackage{wrapfig}
\usepackage{rotating}
\usepackage[normalem]{ulem}
\usepackage{amsmath}
\usepackage{textcomp}
\usepackage{amssymb}
\usepackage{capt-of}
\usepackage{hyperref}
\author{yubao}
\date{\today}
\title{}
\hypersetup{
 pdfauthor={yubao},
 pdftitle={},
 pdfkeywords={},
 pdfsubject={},
 pdfcreator={Emacs 25.1.1 (Org mode 9.0.6)}, 
 pdflang={English}}
\begin{document}

\tableofcontents

\section{Tutorial: how to use emacs}
\label{sec:orgb93ef2e}
\subsection{Frequency used shortcut key and very useful stuff}
\label{sec:org8567fcb}
Manually Defined:
\begin{itemize}
\item open recent files
\end{itemize}
C-x C-r [M-x recentf-open-files]: open recent files
\begin{itemize}
\item How to input code block quickly "<s + TAB"
\end{itemize}
Input "<s" and then type <TAB>
\subsection{Mode and Buffer Introduction}
\label{sec:orge18ab9f}
\begin{itemize}
\item major mode: only one
\item minor mode: 0-n; for Example: tool-bar, scroll bar
\end{itemize}

C-h m: look up all the enabled minor mode
\begin{itemize}
\item help buffer
\end{itemize}
\subsection{control command}
\label{sec:org0bab889}
\begin{itemize}
\item c-g: quit
\item g: refresh the dired view of the current directory to see change
\end{itemize}

\subsection{Getting Help}
\label{sec:org98903d2}
M-x find-fuction
M-x find-variable
C-h k: describe the function a key runs
C-h f: describe a function
C-h m: get mode-specific information, and this cna list the models currently used
\begin{verbatim}
(global-set-key (kbd "C-h C-f") 'find-function)
(global-set-key (kbd "C-h C-v") 'find-variable)
(global-set-key (kbd "C-h C-k") 'find-function-on-key
\end{verbatim}

\subsection{File and  Buffer commands}
\label{sec:orge14e87e}
\begin{itemize}
\item c-x c-f: open a file
\item c-x c-k: kill the buffer reprsenting a file \ldots{}. not deleteing a file

\item C-x s: saves all fiels with a prompt
\item C-x C-s: saves file without a prompt
\item C-s C-w: saves the file with a different name. Askss you for the name

\item c-x b: switch to a different buffer in a window, asks you which buffer to swich to
\item C-x C-b: Switches buffers, but shows you the list of buffers in a new window
\end{itemize}


\begin{itemize}
\item m-<: begining of the buffer
\item m->: end of the file

\item M-x recover-file: recovers the auto-saved file
\item M-x write-file: write the buffer to a different file
\end{itemize}
\subsection{Mouse move command}
\label{sec:org745ba2a}
\subsubsection{word}
\label{sec:org7f30348}
\begin{itemize}
\item c-f: move forward one character
\item c-b: move back one character
\item M-f: move forward one world
\item M-b: move back one world
\end{itemize}
\subsubsection{line}
\label{sec:orgac9fbf8}
\begin{itemize}
\item c-p: previous line/up
\item c-n: next line/down
\item c-a: move to the beginning of the a line
\item c-e: move to the end of the line
\end{itemize}
\subsubsection{page}
\label{sec:orgdf84813}
\begin{itemize}
\item c-v: gage down
\item m-v: page up
\end{itemize}
\subsubsection{screen}
\label{sec:org48b556d}
\begin{itemize}
\item c-l: center the screen
\end{itemize}

\subsection{Edit command}
\label{sec:orgdd5bda0}
\begin{itemize}
\item c-d: delete a character
\item m-d: delete a word

\item c-\_: undo
\item c-/: undo
\item C-g c-/: Redo

\item c-w: cut
\item c-y: yandk/paste

\item m-u: upper case
\item m-l: lower case
\item m-c: capitalize
\end{itemize}

\subsection{Multiple Windos}
\label{sec:org2c70ec6}
\begin{itemize}
\item C-M-v: scroll other window
\item c-x 2: split top/down
\item c-x 3: split left/right
\item c-x o: other window
\end{itemize}

\subsection{search}
\label{sec:org315cc50}
\begin{itemize}
\item c-s text: search
\item c-s TEXT: case sensitive search
\item m-x query-replace <----> m-\%
\item m-x replace-string

\item M-C-s: search a regexp

\item M-s o: searches and shows alll the occurances in an \textbf{Occur} buffer. You can click on the lines to jump to those lines.

\item m-x grep <enter>
\end{itemize}

\subsection{mark}
\label{sec:org38d1dcd}
\begin{itemize}
\item c-space: start/toggle marking a region
\end{itemize}

\subsection{check}
\label{sec:orge4b6fc0}

\begin{itemize}
\item m-\$: spell check word
\item m-x flyspell-mode
\item m-x ispell-region: check a small region
\item m-x ispell-buffer: check all of the buffer
\end{itemize}

\subsection{shell}
\label{sec:orge9a042c}
\begin{itemize}
\item m-x shell: start a bash command line
\end{itemize}

\subsection{Customize variavle, group, mode, function}
\label{sec:orgbfa79ef}

\section{Configuration for project and IDE}
\label{sec:org018c839}
\subsection{Package Management}
\label{sec:orgb7a6a43}
\begin{itemize}
\item Introduction
\end{itemize}
MELPA: Milkypostman's Emacs Lisp Package Archives
\begin{itemize}
\item M-x  package-list-packages
\end{itemize}
d: delete
i: install
x: execute
\begin{itemize}
\item \href{https://melpa.org/}{melpa.org}
\item Add melpa package
\end{itemize}
\begin{verbatim}
  ;;;initialize package
  (require 'package)
  (setq package-archives '(
  ;			 ("gnu" . "https://elpa.gnu.org/packages/")
;			   ("melpa" . "https://melpa.org/packages/")
			   ("melpa-stable" . "https:://stable.melpa.org/packages/"))
  )
  (package-initialize)
\end{verbatim}
\begin{itemize}
\item Auto install package configuration
\begin{verbatim}

(when (>= emacs-major-version 24)
  (require 'package)
  (package-initialize)
  (add-to-list 'package-archives  '(
				    ("melpa" . "https://melpa.org/packages/"))))
(require 'cl)
;;add whatever package you want here
(defvar yubao/packages '(
			 company
			 )
  "Default packages")
(defun yubao/packages-installed-p ()
  (loop for pkg in yubao/packages
	when (not (package-installed-p pkg)) do (return nil)
	finally (return t)))

(unless (yubao/packages-installed-p)
  (message "%s" "Refreshing package database .... ")
  (package-refresh-contents)
  (dolist (pkg yubao/packages)
    (when (not (package-installed-p pkg))
      (package-install pkg))))
\end{verbatim}
\end{itemize}
\subsection{linum Mode}
\label{sec:orga5d2bf8}
(global-linum-mode t)
(linum-mode t)

\subsection{Company Mode}
\label{sec:org9b1370c}
\begin{itemize}
\item \href{http://company-mode.github.io/}{company-mode}
\item What's Company Mode?
\end{itemize}
Company => company anything
\begin{itemize}
\item How to enable company mode?
\end{itemize}

(company-mode t);work on current buffer
(global-company-mode t);work on all the opened buffer

Use M-n or M-p to select candidate item
\subsection{Speedbar}
\label{sec:org2ecdcbf}
m-x speedbar <enter> or m-x speed <tab> <enter> :list project files
\subsection{Compile}
\label{sec:orgbd602fd}
m-x compile
\subsection{Debug}
\label{sec:org05e9f29}
c-x ` : jump to the next error. That ` is a back quote on the top left of the keyboard
\subsection{Format}
\label{sec:org4bcbb1b}
\begin{itemize}
\item Auto Update the Sequence Number
\end{itemize}
Example:
\begin{enumerate}
\item first
\item second
\item third
\item fourth
\end{enumerate}

Then I want to insert one item:   
Example:
\begin{enumerate}
\item first
\item second
\item Inserted new item
\item third
\item fourth
\end{enumerate}

Therefore, think a question: how to auto sort the list?

Method:

Move the curser to the end, and press 'M' (meta), and then press Return key.

Sorted items:
\begin{enumerate}
\item first
\item second
\item Inserted new item
\item third
\item fourth

\item Indent
M-x indent-gegion: indents the region
\end{enumerate}
\subsection{Show match parents "()"}
\label{sec:org97017b6}
[menu]=>[Options]=>[Highlight Matching Parentheses]

\begin{verbatim}
(add-hook 'emacs-lisp-mode-hook 'show-paren-mode)
\end{verbatim}
\subsection{Highlight current line}
\label{sec:orgf946979}
\begin{verbatim}
(global-hl-line-mode t)
\end{verbatim}
\subsection{Disable backup file (*.\textasciitilde{})}
\label{sec:orgb8f56b7}
\begin{verbatim}
;;disable backup file (*.~)
(setq make-backup-files nil)
\end{verbatim}
\subsection{Enable Recent Files}
\label{sec:orgd6b896c}
\begin{verbatim}
(require 'recentf)
(recentf-mode t)
(setq recentf-max-menu-items 25)
;;uncomment this statement if u want to use shortcut key
(global-set-key "\C-x\ \C-r" 'recentf-open-files)
\end{verbatim}
\subsection{Delete Selection Mode}
\label{sec:org2257caf}
\begin{verbatim}
;;add delete selection mode
(delete-selection-mode t)
\end{verbatim}

\subsection{Install Hungary Delete mode}
\label{sec:org5146e9f}
\begin{verbatim}
;;config hungry-delete mode
(require 'hungry-delete)
(global-hungry-delete-mode)
\end{verbatim}
\subsection{Install a Theme}
\label{sec:org1f181fb}
\begin{verbatim}
(load-theme 'monokai t)
\end{verbatim}

\begin{verbatim}
(require 'smex) ; Not needed if you use package.el
(global-set-key (kbd "M-x") 'smex)
(global-set-key (kbd "M-X") 'smex-major-mode-commands)
;; This is your old M-x.
(global-set-key (kbd "C-c C-c M-x") 'execute-extended-command)
\end{verbatim}
\subsection{Install swiper and counsel}
\label{sec:org7a0dca9}
\begin{itemize}
\item \href{https://github.com/abo-abo/swiper}{swiper}
\item configuration  
\begin{verbatim}
  (ivy-mode 1)
(setq ivy-use-virtual-buffers t)
(setq enable-recursive-minibuffers t)
(global-set-key "\C-s" 'swiper)
(global-set-key (kbd "C-c C-r") 'ivy-resume)
(global-set-key (kbd "<f6>") 'ivy-resume)
(global-set-key (kbd "M-x") 'counsel-M-x)
(global-set-key (kbd "C-x C-f") 'counsel-find-file)
(global-set-key (kbd "<f1> f") 'counsel-describe-function)
(global-set-key (kbd "<f1> v") 'counsel-describe-variable)
(global-set-key (kbd "<f1> l") 'counsel-find-library)
(global-set-key (kbd "<f2> i") 'counsel-info-lookup-symbol)
(global-set-key (kbd "<f2> u") 'counsel-unicode-char)
(global-set-key (kbd "C-c g") 'counsel-git)
(global-set-key (kbd "C-c j") 'counsel-git-grep)
(global-set-key (kbd "C-c k") 'counsel-ag)
(global-set-key (kbd "C-x l") 'counsel-locate)
(global-set-key (kbd "C-S-o") 'counsel-rhythmbox)
(define-key read-expression-map (kbd "C-r") 'counsel-expression-history)
\end{verbatim}
\end{itemize}
\subsection{Install and Configure Smartparens mode}
\label{sec:orgc3631cc}
\begin{itemize}
\item Install
\href{https://github.com/Fuco1/smartparens\#getting-started}{samartparents}
\item Configure
\end{itemize}
\begin{verbatim}
(require 'smartparens-config)
(add-hook 'emacs-lisp-mode-hook 'smartparens-mode)
\end{verbatim}
\subsection{Configure Javascript IDE}
\label{sec:org7a7ca8e}
\begin{itemize}
\item install js2-mode in Emacs
\item Configuration js2-mode
\end{itemize}
Thde default mode is "javascript mode", use this to change to Javascript IDE:
\begin{verbatim}
;;configure for js2-mode
(setq auto-mode-alist
      (append
       '(("\\.js\\'" . js2-mode))
auto-mode-alist))
\end{verbatim}
\begin{itemize}
\item Install nodejs in OS
\item Install nodejs-repl
\item Configure nodejs-repl
set nodejs-repl-command to "nodejs" in ubuntu system
\end{itemize}
\begin{verbatim}
;Type M-x nodejs-repl to run Node.js REPL. See also comint-mode to check key bindings.
;You can define key bindings to send JavaScript codes to REPL like below:

(add-hook 'js-mode-hook
	  (lambda ()
	    (define-key js-mode-map (kbd "C-x C-e") 'nodejs-repl-send-last-sexp)
	    (define-key js-mode-map (kbd "C-c C-r") 'nodejs-repl-send-region)
	    (define-key js-mode-map (kbd "C-c C-l") 'nodejs-repl-load-file)
	    (define-key js-mode-map (kbd "C-c C-z") 'nodejs-repl-switch-to-repl)))	    

\end{verbatim}
\section{org-mode basics}
\label{sec:orga698717}
\subsection{Introduction and Common Configuration}
\label{sec:org80d1339}
\subsubsection{How to enter source code edit mode}
\label{sec:org373ad55}
\begin{itemize}
\item C-c ' (C-c and single quote) to enter into the source code edit mode, and then use it to turn back
\item C-c C-k to abort
\item Example:
\end{itemize}
\begin{verbatim}
;;press "C-c ' " to edit source code
  (message "Emacs lisp")
\end{verbatim}
\subsection{Schedule and Calenda and Todo}
\label{sec:orga95cd24}
\begin{itemize}
\item TODO creating todo and donw items
\end{itemize}

shift-RightArrow or C-c C-t: togle TODO state
\begin{itemize}
\item C-c C-s: to schedule time
\item C-c C-d: to set deadline of time
\item C-c a: lookup the schedual
\end{itemize}
\subsubsection{{\bfseries\sffamily TODO} todo}
\label{sec:orgb086bde}
\subsubsection{{\bfseries\sffamily DONE} done}
\label{sec:org54cf4bd}
\subsection{Links}
\label{sec:orge5f1d9b}
\begin{itemize}
\item \href{http:www.baidu.com}{baidu} : www.baidu.com  C-c C-l: edit the link
\end{itemize}

\section{Emacs Lisp}
\label{sec:org46b4aba}
\subsection{Study Resources}
\label{sec:org78139e1}
\begin{itemize}
\item \href{https://learnxinyminutes.com/}{learnxinyminutes}
\end{itemize}
\subsection{Command}
\label{sec:orgacdecc6}
\begin{itemize}
\item M-: -> :to go to the evaluate buffer where you can evaluate a lisp statement.
\end{itemize}
For example, "setq" sets a variable to a value: (setq your$\backslash$\(_{\text{var}}\) '123)

\begin{itemize}
\item M-x ielm : ELISP, describe-mode for help
\item C-x C-e:  runs the command eval-last-sexp (found in global-map)
\item M-x eval-buffer :run commands on the current buffer
\end{itemize}
\subsection{Elisp Grammer}
\label{sec:org316d97f}
\subsubsection{Example}
\label{sec:orgae17ab6}
\begin{verbatim}
;;set a variable
(setq my-name "yubao")

;;show the variable's value
(message my-name)

;;define a func to show my name
(defun showMyName ()
(interactive);M-x call
(message "Hello, %s" my-name)
)

;;call "showMyName" fuction
(showMyName)

;;how to bind the key
(global-set-key (kbd "<C-f2>") 'showMyName)

\end{verbatim}

\subsubsection{Variable}
\label{sec:org0729716}

\subsubsection{Function}
\label{sec:orgb22abc7}

   (+ 2 2)
p
\section{Reference}
\label{sec:org8a263f2}
\begin{itemize}
\item \href{http://book.emacs-china.org/}{MasterEmacsIn21Days}
\item \href{https://learnxinyminutes.com/}{learnxinyminutes}
\end{itemize}
\end{document}